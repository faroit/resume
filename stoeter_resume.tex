%-------------------------
% Resume in Latex
% Author : Sourabh Bajaj, Fabian-Robert Stöter
% License : MIT
%------------------------

\documentclass[a4paper,11pt]{article}
\usepackage[utf8]{inputenc}
\usepackage{latexsym}
\usepackage[empty]{fullpage}
\usepackage{titlesec}
\usepackage{marvosym}
\usepackage[usenames,dvipsnames]{color}
\usepackage{verbatim}
\usepackage{enumitem}
\usepackage[hidelinks]{hyperref}
\usepackage{fancyhdr}
\usepackage[english]{babel}

% fix for pandoc 1.14
\providecommand{\tightlist}{%
  \setlength{\itemsep}{0pt}\setlength{\parskip}{0pt}}

% \usepackage[sfdefault]{FiraSans} %% option 'sfdefault' activates Fira Sans as the default text font
% \usepackage[T1]{fontenc}
% \renewcommand*\oldstylenums[1]{{\firaoldstyle #1}}
\pagestyle{fancy}
\fancyhf{} % clear all header and footer fields
\fancyfoot{}
\renewcommand{\headrulewidth}{0pt}
\renewcommand{\footrulewidth}{0pt}

% Adjust margins
\addtolength{\oddsidemargin}{-0.5in}
\addtolength{\evensidemargin}{-0.5in}
\addtolength{\textwidth}{1in}
\addtolength{\topmargin}{-.5in}
\addtolength{\textheight}{1.0in}

\urlstyle{same}

\raggedbottom
\raggedright
\setlength{\tabcolsep}{0in}

% Sections formatting
\titleformat{\section}{
  \vspace{-4pt}\scshape\raggedright\large
}{}{0em}{}[\color{black}\titlerule \vspace{-5pt}]

%-------------------------
% Custom commands
\newcommand{\resumeItem}[2]{
  \item\small{
    \textbf{#1}{: #2 \vspace{-2pt}}
  }
}

\newcommand{\resumeSubheading}[4]{
  \vspace{-1pt}\item
    \begin{tabular*}{0.97\textwidth}[t]{l@{\extracolsep{\fill}}r}
      \textbf{#1} & #2 \\
      \textit{\small#3} & \textit{\small #4} \\
    \end{tabular*}\vspace{-5pt}
}

\newcommand{\resumeSubItem}[2]{\resumeItem{#1}{#2}\vspace{-4pt}}

\renewcommand{\labelitemii}{$\circ$}

\newcommand{\resumeSubHeadingListStart}{\begin{itemize}[leftmargin=*]}
\newcommand{\resumeSubHeadingListEnd}{\end{itemize}}
\newcommand{\resumeItemListStart}{\begin{itemize}}
\newcommand{\resumeItemListEnd}{\end{itemize}\vspace{-5pt}}

%-------------------------------------------
%%%%%%  CV STARTS HERE  %%%%%%%%%%%%%%%%%%%%%%%%%%%%


\begin{document}

%----------HEADING-----------------
\begin{tabular*}{\textwidth}{l@{\extracolsep{\fill}}r}
  \textbf{\href{https://faroit.com}{\Large Dr.-Ing. Fabian-Robert St\"oter}} & Email : \href{mailto:mail@faroit.com}{mail@faroit.com}\\
  \href{https://faroit.com}{https://faroit.com} & Mobile : +33637184935 \\
\end{tabular*}


%-----------EDUCATION-----------------
\section{Education}
  \resumeSubHeadingListStart
    \resumeSubheading
      {Fraunhofer IIS and University of Erlangen-Nürnberg}{Germany}
      {PhD/ Dr.-Ing. (Summa cum laude / Primi Ordinis)}{2019}
      \resumeItemListStart
        \resumeItem{PhD Thesis}
          {``Separation and Count Estimation for Audio Sources Overlapping in Time and Frequency'', International Audio Laboratories Erlangen, Examiners: Prof. Dr.-Ing. Bernd Edler (AudioLabs), Prof. Gaël Richard PhD. (ParisTech)}
      \resumeItemListEnd
    \resumeSubheading
      {University of Hannover}{Germany}
      {Diplom Ingenieur/Master of Science in Electrical Engineering.}{2012}
    \resumeItemListStart
      \resumeItem{Master thesis}
        {``Low Delay Error Concealment for Audio signals'', Institute for Information Processing, Examiner: Prof. Dr.-Ing Jörn Ostermann.}
      \resumeItem{Study Project/Bachelor Thesis}
        {``Image Segmentation Using Graph-Cuts'', Institute for Information Processing in Hannover, Examiner: Prof. Dr. Bodo Rosenhahn.\\
        }
    \resumeItemListEnd

  \resumeSubHeadingListEnd

%-----------EXPERIENCE-----------------
\section{Experience}
  \resumeSubHeadingListStart

    \resumeSubheading
    {Audioshake.ai}{San Francisco, USA}
    {Head of Research}{Present}
    \resumeItemListStart
      \resumeItem{Leading the research team}
        {Creating state-of-the-art music separation and tagging models.}
      \resumeItemListEnd

    \resumeSubheading
    {Sony Europe, Sony Japan}{}
    {Visiting Researcher}{Summer 2021}
    \resumeItemListStart
      \resumeItem{Self-supervised learning}
        {Representation learning for audio. Submission to HEAR challenge.}
      \resumeItem{Workshop Chairing}
        {Organization and chairing of Music Demixing Challenge (MDX) and Workshop.}
    \resumeItemListEnd

    \resumeSubheading
      {Inria and LIRMM}{Montpellier, France}
      {Postdoctoral Researcher / Research Engineer}{2017 - 2021}
      \resumeItemListStart
        \resumeItem{Kamoulox Project (French ANR)}
          {Research on AI based audio restoration techniques on big data.}
        \resumeItem{COS4CLOUD/Pl@ntnet (EU Horizon 2020)}
          {computer-vision AI research on biodiversity data.}
        \resumeItemListEnd

    \resumeSubheading
      {International Audio Laboratories Erlangen}{Erlangen, Germany}
      {Research and Teaching Assistant}{2012 - 2017}
      \resumeItemListStart
        \resumeItem{Research Interests}
          {audio signal processing, machine learning and deep learning, signal representations, auditory filter banks and modulations, perceptual evaluation for music and speech, speaker count estimation.}
        \resumeItem{Teaching}
          {
          \emph{Seminars}: Reproducible Audio Research Seminar. 
          \emph{Courses}: Statistical Methods for Audio Experiments in R and Python (2013-2016)
          }
        \resumeItemListEnd

    \resumeSubheading
      {Dolby Inc.}{Nuremberg, Germany}
      {Engineering Intern: Testing and Productization, Professional Licensing}{2011}

    \resumeSubheading
      {FAROIT.COM}{Germany}
      {Entrepreneurship: Web Services for various PR-companies, private banks and marketing companies.}{2001 - 2014}

  \resumeSubHeadingListEnd

%-----------PROJECTS-----------------
\section{Fully Supervision of students and early-stage researchers}
  \resumeSubHeadingListStart
    \resumeSubItem{\textit{Michael Tänzer}}
    {PhD student, Fraunhofer IDMT, Germany, Internship on audio tagging (Summer 2021)}
    \resumeSubItem{\textit{Lucas Mathieu}}
    {Master student, AgroParistech (France), Master thesis ``Listening to the Wild'' (03/2020)}
    \resumeSubItem{\textit{Clara Jacintho, Delton Vaz}}
    {Bachelor, PolyTech Montpellier (France), Research Project, ``Machine Learning for Audio on the Web'' (12/2019)}
    \resumeSubItem{\textit{Wolfgang Mack}}
    {Master student, University of Erlangen-Nuremberg (Germany), Master thesis ``Investigations on Speaker Separation using Embeddings obtained by Deep Learning'' (05/2017)}
    \resumeSubItem{\textit{Nils Werner}}
    {Master student, University of Erlangen-Nuremberg (Germany, Master thesis titled ``Parameter Estimation for Time-Varying Harmonic Audio Signals'' (02/2014)}
    \resumeSubItem{\textit{Jeremy Hunt}}
    {DAAD funded Scholarship, Rice University (USA), Research Internship, ``Fast implementation of ND Non-Negative-Tensor Factorisation'' (04/2016)}
    \resumeSubItem{\textit{Erik Johnson}}
    {DAAD funded Scholarship, Carleton University (Canada), Research Internship, ``Open-Source Implementation of Multichannel BSSEval in Python'' (03/2014)}
    \resumeSubItem{\textit{Berkan Ercan}}
    {Master student, Bilkent University (Turkey), Master thesis ``Music Instrument Source Separation'' (03/2013)}
  \resumeSubHeadingListEnd

\section{Academic Resources and Challenges}
  \resumeSubHeadingListStart
  \resumeSubItem{Chair}{\textbf{2021 Music Demixing Challenge} in collaboration with Sony Inc. Japan. Challenge targeted machine learning researchers and was hosted on \href{https://www.aicrowd.com/challenges/music-demixing-challenge-ismir-2021}{AICrowd}. Challenge received over 1500 submissions.}
  \resumeSubItem{Task Leader}{\textbf{Professionally Produced Music Demixing} for SiSEC (Signal Separation Evaluation Campaign) 2016 and 2018)
  SiSEC is the reference international challenge for signal processing researchers to benchmark their methods.}
  \resumeSubItem{Task co-chair}{\textbf{BirdCLEF challenge} since 2019, happening in the context of the CLEF conference and the SIGIR.}
  \resumeSubItem{Co-Organizer}{\textbf{LifeClef challenge} since 2020, happening in the context of the CLEF conference and the SIGIR.}
  \resumeSubItem{Principal coordinator}{\textbf{sigsep}, (\url{https://sigsep.github.io/}), a popular resource for researchers in source separation.
  It comprises one of the most popular music datasets in academia – \textsc{MUSDB18} – as well as many software tools and teaching resources.}
  \resumeSubHeadingListEnd

\section{Editing and Chairing}
\resumeSubHeadingListStart  
  \resumeSubItem{Technology-Chair}{International Conference on Music Information Retrieval (ISMIR 21)}
  \resumeSubItem{Program Chair and General Co-Chair}{``2021 Music Demixing Workshop'': \url{https://mdx-workshop.github.io} with over 150 participants}
  \resumeSubItem{Topic Editor}{Audio Machine Learning for the  ``Journal of Open Source Software''}
\resumeSubHeadingListEnd

\section{Reviewing}
\resumeSubHeadingListStart
  \resumeSubItem{Journals}{``Transaction in Audio, Speech and Language Processing'', ``Selected Topics in Signal Processing'' and ``Signal Processing Letters'', and of ``EURASIP'' journal since 2015. Furthermore I am reviewing for the ``Journal of Open Source Software''.}
  \resumeSubItem{Conferences}{ICASSP, EUSIPCO, DAFx, ISMIR}
\resumeSubHeadingListEnd

\section{International Collaborations}
\resumeSubHeadingListStart
  \resumeSubItem{Zafar Rafii}{(Audible Magic, USA, senior scientist). Community service for researchers working on audio source separation.}
  \resumeSubItem{Roland Badeau}{(Telecom Paris, France, prof.). Probabilistic models for signal processing.}
  \resumeSubItem{Antoine Liutkus}{(Inria, France, CRCN). Probabilistic models for signal processing and optimal transport, leading to the conference papers.}
  \resumeSubItem{Mark Plumbley}{(Univ. Surrey, UK, prof.). Perceptual evaluation for large scale audio processing.}
  \resumeSubItem{Yuki Mitsufuji}{(Sony, Japan, Deputy manager of science). Baseline system for music demixing and international challenge.}
\resumeSubHeadingListEnd

\section{Graduate programs}
 I gave over 120 hours of classes between 2014 and 2016 in the, divided into practical work on computers and tutorials. I also taught distinct subjects in computer science and signal processing:
\resumeSubHeadingListStart
 \resumeSubItem{Guest Lecture (2h)}{2021, ``Music Source Separation'', class by Meinard Müller and Emanuel Habets, University of Erlangen-Nürnberg, Germany}
 \resumeSubItem{Introduction to Deep Learning (6h)}{2018, 2019, Master 2, PolyTech Montpellier}
 \resumeSubItem{Reproducible Audio Research Seminar (12h)}{(2016), international master in signal processing, University of Erlangen (Germany)}
 \resumeSubItem{Statistical Methods for Audio Experiments}{2013-2016, (60h), international master in signal processing, University of Erlangen (Germany).}
\resumeSubHeadingListEnd

\section{Selected Invited Talks/Tutorials/Summer Schools}
  \resumeSubHeadingListStart
    \resumeSubItem{Invited Talk (1h) at AES Virtual Symposium: Applications of Machine Learning in Audio}{
       "Current Trends in Audio Source Separation". (\href{https://www.youtube.com/watch?v=AB-F2JmI9U4}{Video})}
    \resumeSubItem{Tutorial (3h)}
       {»Music Separation with DNNs: Making It Work«, ISMIR, 2018 (Paris, September 4th, 2018)}
    \resumeSubItem{Invited Talk (1h)}
    {»Deep Learning for Music Unmixing«, Deep Learning: from
      theory to applications, Technicolor (Rennes, September 23th, 2018)}
  \resumeSubHeadingListEnd

\section{Prizes/Awards/Press}
\resumeSubHeadingListStart
    \resumeSubItem{Interview}{Deutschlandfunk (German public radio): \href{https://www.deutschlandfunkkultur.de/recycling-von-songs-wie-ki-neue-musik-generiert-dlf-kultur-90e01124-100.html}{Recycling von Songs: Wie KI neue Musik generiert}, 12/2021}
    \resumeSubItem{Global summer PyTorch Hackathon 2020}
        {Winner with DeMask~\href{https://pytorch.org/blog/announcing-the-winners-of-the-2020-global-pytorch-summer-hackathon/}{See PyTorch Website}}
    \resumeSubItem{Global summer PyTorch Hackathon 2019}
        {Winner with Open-Unmix~\href{https://anr.fr/fr/actualites-de-lanr/details/news/open-unmix-un-logiciel-open-source-issu-du-projet-anr-kamoulox-pour-demixer-la-musique/}{See ANR Website}}
\resumeSubHeadingListEnd

% --------PROGRAMMING SKILLS------------
\section{Skills and Tools}
I'm an active open-source contributor, most of my projects are based in Python: \url{https://github.com/faroit/}
 \resumeSubHeadingListStart
   \resumeSubItem{Programming Languages}
     {Python, Javascript, Matlab, R, C, Java, Julia, SQL}
   \resumeSubItem{ML-Ops/Deployment Tools}
     {Docker, tf-serving, ONNX, tfjs, AWS, HPC}
   \resumeSubItem{Machine Learning Frameworks}
     {PyTorch, Tensorflow, Keras, NNabla, Scikit-Learn, statsmodels, pandas}
     \resumeSubHeadingListEnd
     
\section{Selected Peer-Reviewed Publications}

\begin{itemize}
\tightlist
\item
  Y. Mitsufuji, G. Fabbro, S. Uhlich, \textbf{F.-R. Stöter} et al., “Music Demixing Challenge 2021,” Journal, Frontiers in Signal Processing, vol. 1, 2022
\item 
  M. Pariente, S. Cornell, J. Cosentino, S. Sivasankaran, E. Tzinis, J. Heitkaemper, M. Olvera, \textbf{F.-R. Stöter}, et al. “Asteroid: the PyTorch-based audio source separation toolkit for researchers”,  Proc. Interspeech. 2020.
\item
  \textbf{F.-R. Stöter}, S. Uhlich, A. Liutkus, and Y. Mitsufuji.
  \emph{Open-Unmix - A Reference Implementation for Music Source Separation}, Journal of Open Source Software, 2019
\item
  A. Liutkus, U. Simsekli, S. Majewski, A. Durmus, and \textbf{F.-R. Stöter}.
  \emph{Sliced-Wasserstein Flows: Nonparametric Generative Modeling via
  Optimal Transport and Diffusions.}, Proc. ICML 2019.
\item
  E. Cano, D. FitzGerald, A. Liutkus, MD. Plumbley, and \textbf{F.-R. Stöter}.
  \emph{Musical Source Separation: An Introduction} IEEE Signal
  Processing Magazine, 36, 2019.
\item
  \textbf{F.-R. Stöter}, S. Chakrabarty, B. Edler, and E. A. P. Habets.
  \emph{CountNet: Estimating the Number of Concurrent Speakers Using
  Supervised Learning.} IEEE/ACM Transactions on Audio, Speech, and
  Language Processing, Nov.~2018.
\item
  \textbf{F.-R. Stöter}, S Chakrabarty, B. Edler, and E.A.P. Habets.
  \emph{Classification vs.~Regression in Supervised Learning for Single
  Channel Speaker Count Estimation.} Proc. ICASSP, 2018.
\item
  Z. Rafii, A. Liutkus, \textbf{F.-R. Stöter}, S. I. Mimilakis, D.
  FitzGerald, and B. Pardo. \emph{An Overview of Lead and Accompaniment
  Separation in Music.} In: IEEE/ACM Transactions on Audio, Speech, and
  Language Processing, Aug.~2018.
\item
  \textbf{F.-R. Stöter}, A. Liutkus, and N. Ito, \emph{The 2018 Signal
  Separation Evaluation Campaign.} International Conference on Latent
  Variable Analysis and Signal Separation, 2018.
\item
  M. Schoeffler, S. Bartoschek, \textbf{F.-R. Stöter}, M. Roess, S.
  Westphal, B. Edler and J. Herre, \emph{webMUSHRA---A Comprehensive
  Framework for Web-based Listening Tests} Journal of Open Research
  Software, 2018.
\item
  A. Liutkus, \textbf{F.-R. Stöter}, Z. Rafii, D. Kitamura, B. Rivet, N.
  Ito, N. Ono, and J. Fontecave \emph{The 2016 Signal Separation Evaluation
  Campaign} Proc. of Latent Variable Analysis and Signal Separation, Grenoble, France, 2017.
\item
  \textbf{F.-R. Stöter}, A. Liutkus, R. Badeau, B. Edler, and P. Magron
  \emph{Common Fate Model for Unison Source Separation} Proc. ICASSP, 
  Shanghai, China, 2016.
\item
  M. Schoeffler, \textbf{F.-R. Stöter}, B. Edler, and J. Herre \emph{Towards
  the next generation of web-based experiments: a case study assessing
  basic audio quality following the ITU-R BS1534
  (MUSHRA)} 1st Web Audio Conference, Paris, France, 2016.
\item
  \textbf{F.-R. Stöter}, M. Müller, and B. Edler
  \emph{Multi-Sensor Cello Recordings for Instantaneous Frequency
  Estimation.} Proc. of the ACM Multimedia, Brisbane, Australia, 2015.
\item
  \textbf{F.-R. Stöter}, N. Werner, S. Bayer, and B. Edler \emph{Refining
  Fundamental Frequency Estimates using Time Warping.} Proc. of
  EUSIPCO, Nice, France, 2015.
\item
  \textbf{F.-R. Stöter}, S. Bayer, and B. Edler \emph{Unison Source
  Separation} Proc. DAFx, Erlangen, Germany, 2014.
\item
  \textbf{F.-R. Stöter}, M. Schoeffler, B. Edler and J. Herre \emph{Human
  ability of counting the number of instruments in polyphonic music.}
  Proc. ICA, Montreal, Canada, 2013
\item
  M. Schoeffler, \textbf{F.-R. Stöter}, H. Bayerlein, B. Edler and J. Herre
  \emph{An experiment about estimating the number of instruments in
  polyphonic music: a comparison between internet and laboratory
  results.} Proc. ISMIR, Curitiba, Brazil, 2013.
\end{itemize}

% --------Language SKILLS------------
\section{Languages}
\resumeSubHeadingListStart
  \item{
    \textbf{German}{: native}
    \textbf{English}{: fluent (C1)}
    \textbf{French}{: basic (B1)}
  }
\resumeSubHeadingListEnd

\end{document}
