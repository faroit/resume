%-------------------------
% Resume in Latex
% Author : Sourabh Bajaj, Fabian-Robert Stöter
% License : MIT
%------------------------

\documentclass[a4paper,11pt]{article}
\usepackage[utf8]{inputenc}
\usepackage{latexsym}
\usepackage[empty]{fullpage}
\usepackage{titlesec}
\usepackage{marvosym}
\usepackage[usenames,dvipsnames]{color}
\usepackage{verbatim}
\usepackage{enumitem}
\usepackage[hidelinks]{hyperref}
\usepackage{fancyhdr}
\usepackage[english]{babel}

% fix for pandoc 1.14
\providecommand{\tightlist}{%
  \setlength{\itemsep}{0pt}\setlength{\parskip}{0pt}}

% \usepackage[sfdefault]{FiraSans} %% option 'sfdefault' activates Fira Sans as the default text font
% \usepackage[T1]{fontenc}
% \renewcommand*\oldstylenums[1]{{\firaoldstyle #1}}
\pagestyle{fancy}
\fancyhf{} % clear all header and footer fields
\fancyfoot{}
\renewcommand{\headrulewidth}{0pt}
\renewcommand{\footrulewidth}{0pt}

% Adjust margins
\addtolength{\oddsidemargin}{-0.5in}
\addtolength{\evensidemargin}{-0.5in}
\addtolength{\textwidth}{1in}
\addtolength{\topmargin}{-.5in}
\addtolength{\textheight}{1.0in}

\urlstyle{same}

\raggedbottom
\raggedright
\setlength{\tabcolsep}{0in}

% Sections formatting
\titleformat{\section}{
  \vspace{-4pt}\scshape\raggedright\large
}{}{0em}{}[\color{black}\titlerule \vspace{-5pt}]

%-------------------------
% Custom commands
\newcommand{\resumeItem}[2]{
  \item\small{
    \textbf{#1}{: #2 \vspace{-2pt}}
  }
}

\newcommand{\resumeSubheading}[4]{
  \vspace{-1pt}\item
    \begin{tabular*}{0.97\textwidth}[t]{l@{\extracolsep{\fill}}r}
      \textbf{#1} & #2 \\
      \textit{\small#3} & \textit{\small #4} \\
    \end{tabular*}\vspace{-5pt}
}

\newcommand{\resumeSubItem}[2]{\resumeItem{#1}{#2}\vspace{-4pt}}

\renewcommand{\labelitemii}{$\circ$}

\newcommand{\resumeSubHeadingListStart}{\begin{itemize}[leftmargin=*]}
\newcommand{\resumeSubHeadingListEnd}{\end{itemize}}
\newcommand{\resumeItemListStart}{\begin{itemize}}
\newcommand{\resumeItemListEnd}{\end{itemize}\vspace{-5pt}}

%-------------------------------------------
%%%%%%  CV STARTS HERE  %%%%%%%%%%%%%%%%%%%%%%%%%%%%


\begin{document}

%----------HEADING-----------------
\begin{tabular*}{\textwidth}{l@{\extracolsep{\fill}}r}
  \textbf{\href{https://www.faroit.com}{\Large Dr.-Ing. Fabian-Robert St\"oter}} & Email : \href{mailto:mail@faroit.com}{mail@faroit.com}\\
  \href{https://www.faroit.com}{https://www.faroit.com} & Mobile : +33637184935 \\
\end{tabular*}


%-----------EDUCATION-----------------
\section{Education}
  \resumeSubHeadingListStart
    \resumeSubheading
      {Fraunhofer IIS and University of Erlangen-Nürnberg}{Germany}
      {PhD/ Dr.-Ing. (Summa cum laude)}{2019}
      \resumeItemListStart
        \resumeItem{PhD Thesis}
          {``Separation and Count Estimation for Audio Sources Overlapping in Time and Frequency'', International Audio Laboratories Erlangen, Examiners: Prof. Dr.-Ing. Bernd Edler (AudioLabs), Prof. Gaël Richard PhD. (ParisTech)}
      \resumeItemListEnd
    \resumeSubheading
      {University of Hannover}{Germany}
      {Diplom Ingenieur/Master of Science in Electrical Engineering.}{2012}
    \resumeItemListStart
      \resumeItem{Master thesis}
        {``Low Delay Error Concealment for Audio signals'', Institute for Information Processing, Examiner: Prof. Dr.-Ing Jörn Ostermann.}
      \resumeItem{Study Project/Bachelor Thesis}
        {``Image Segmentation Using Graph-Cuts'', Institute for Information Processing in Hannover, Examiner: Prof. Dr. Bodo Rosenhahn.\\
        }
    \resumeItemListEnd

    % \resumeSubheading
    %   {University of Hannover}{Germany}
    %   {Industrial Engineering.}{2003 -- 2005}
    % \resumeSubheading
    %   {Friedrich-Ebert Gymnasium}{Hamburg, Germany}
    %   {Abitur}{2002}
  \resumeSubHeadingListEnd

%-----------EXPERIENCE-----------------
\section{Experience}
  \resumeSubHeadingListStart

    \resumeSubheading
    {Audioshake.ai}{Montpellier, France}
    {Head of Research}{Present}
    \resumeItemListStart
      \resumeItem{Music Separation}
        {Leading the research team of AI based state-of-the-art music separation tech.}
      \resumeItemListEnd

    \resumeSubheading
    {Sony Europe, Sony Japan}{}
    {Visiting Researcher}{Summer 2021}
    \resumeItemListStart
      \resumeItem{Self-Supervised learning}
        {Speech and music AI}
      \resumeItemListEnd

    \resumeSubheading
      {Inria and LIRMM}{Montpellier, France}
      {Research Engineer}{2017 - 2021}
      \resumeItemListStart
        \resumeItem{Kamoulox Project (French National Funding)}
          {Research of AI based audio restoration techniques on big data.
           Development of open-source implementation of source separation and enhancement (Open-Unmix).
          }
        \resumeItem{COS4CLOUD/Pl@ntnet (EU Horizon 2020)}
          {Pl@ntnet: computer-vision AI research on biodiversity citizen science data.}
        \resumeItemListEnd

    \resumeSubheading
      {International Audio Laboratories Erlangen}{Erlangen, Germany}
      {Research and Teaching Assistant}{2012 - 2017}
      \resumeItemListStart
        \resumeItem{Research Interests}
          {audio signal processing, machine learning and deep learning, signal representations, auditory filter banks and modulations, perceptual evaluation for music and speech, speaker count estimation.}
        \resumeItem{Teaching}
          {\\
          \emph{Seminars}: Reproducible Audio Research Seminar. \\
          \emph{Courses}: Statistical Methods for Audio Experiments in R (2013-2016) \\
          }
        \resumeItemListEnd

    \resumeSubheading
      {Dolby Inc.}{Nuremberg, Germany}
      {Engineering Intern: Testing and Productization, Professional Licensing}{2011}

    \resumeSubheading
      {FAROIT.COM}{Germany}
      {Entrepreneurship: Web Services for various PR-companies, private banks and marketing companies.}{2001 - 2014}

  \resumeSubHeadingListEnd

%-----------PROJECTS-----------------
\section{Supervision of students and early-stage researchers}
  \resumeSubHeadingListStart
    \resumeSubItem{Full supervision of \textit{Lucas Mathieu}}
    {Master 2, AgroParistech (France), Master thesis ``Listening to the Wild'' (03/2020). Theoretical research on self-supervised learning using data from animal-born loggers (MUSE project)}
    \resumeSubItem{Full supervision of \textit{Clara Jacintho, Delton Vaz}}
    {Bachelor, PolyTech Montpellier (France), Research Project, ``Machine Learning for Audio on the Web'' (12/2019). Research on web based separation architectures. Resulted in a paper submitted to the Web Audio Conference 2021}
    \resumeSubItem{Full supervision of \textit{Wolfgang Mack}}
    {Master student, University of Erlangen-Nuremberg (Germany), Master thesis ``Investigations on Speaker Separation using Embeddings obtained by Deep Learning'' (05/2017), Student was accepted as PhD student after master thesis.}
    \resumeSubItem{Full supervision of \textit{Nils Werner}}
    {Master student, University of Erlangen-Nuremberg (Germany, Master thesis titled ``Parameter Estimation for Time-Varying Harmonic Audio Signals'' (02/2014). Student was accepted as PhD student after master thesis.}
    \resumeSubItem{Full supervision of \textit{Berkan Ercan}}
    {Master student, Bilkent University (Turkey), Master thesis ``Music Instrument Source Separation'' (03/2013)}
  \resumeSubHeadingListEnd

\section{Supervision of technological development (Software)}
  \resumeSubHeadingListStart
    \resumeSubItem{Full supervision of \textit{Jeremy Hunt}}
    {DAAD Scholarship (6 months), Rice University, Texas, Research Internship (6 months) ``Fast implementation of ND Non-Negative-Tensor Factorisation'' (04/2016)}
    \resumeSubItem{Full supervision of \textit{Erik Johnson}}
    {Master Student, Carleton University (Canada), Research Internship (6 months) ``Open-Source Implementation of Multichannel BSSEval in Python'' (03/2014)}
  \resumeSubHeadingListEnd

\section{Resources for Researchers}
\resumeSubHeadingListStart
  \item{I am the principal coordinator of \textbf{sigsep}, (\url{https://sigsep.github.io/}), which is a popular resource for researchers in source separation.}
  It comprises the very popular \textsc{MUSDB18} music database, as well as many software tools and teaching resources.
  \item{I was the leader of the task \textbf{Professionally Produced Music Demixing} for SiSEC (Signal Separation Evaluation Campaign) since 2016}
  SiSEC is the reference international challenge for signal processing researchers to benchmark their methods. I have been particularly active in providing the community with automated tools to submit and evaluate their work.
  \item{I was the task co-chair for the \textbf{BirdCLEF challenge} since 2019, happening in the context of the CLEF conference and the special interest group on information retrieval (SIGIR).}
  \item{I was co-organizing part of the \textbf{LifeClef challenge} since 2020, happening in the context of the CLEF conference and the special interest group on information retrieval (SIGIR).}
\resumeSubHeadingListEnd

\section{Challenges}
  \resumeSubHeadingListStart
    \item{I was the co-chair of the \textbf{2021 Music Demixing Challenge} in collaboration with Sony Inc. Japan. The challenge targeted machine learning researchers and was hosted on aicrowd (\url{https://www.aicrowd.com/challenges/music-demixing-challenge-ismir-2021}), a crowd platform for ml challenges. The challenge received over 1500 submissions}
    \item{I was the leader of the task \textbf{Professionally Produced Music Demixing} for SiSEC (Signal Separation Evaluation Campaign) since 2016\\
    SiSEC is the reference international challenge for signal processing researchers to benchmark their methods. I have been particularly active in providing the community with automated tools to submit and evaluate their work.}
    \item{I was the task co-chair for the \textbf{BirdCLEF challenge} since 2019, happening in the context of the CLEF conference and the special interest group on information retrieval (SIGIR).}
    \item{I was co-organizing part of the \textbf{LifeClef challenge} since 2020, happening in the context of the CLEF conference and the special interest group on information retrieval (SIGIR).}
  \resumeSubHeadingListEnd

\section{Editing and Chairing}
\resumeSubHeadingListStart  
\resumeSubItem{Topic Editor}{for Audio Machine Learning for the  ``Journal of Open Source Software'' since 07/21.}
  \resumeSubItem{Tech-Chair}{for the International Conference on Music Information Retrieval (ISMIR 21)}
  \resumeSubItem{Program Chair and General Co-Chair}{for the ``2021 Music Demixing Workshop'': \url{https://mdx-workshop.github.io}}
\resumeSubHeadingListEnd

\section{Reviewing}
\resumeSubHeadingListStart
  \resumeSubItem{Journals}{``Transaction in Audio, Speech and Language Processing'', ``Selected Topics in Signal Processing'' and ``Signal Processing Letters'', and of ``EURASIP'' journal since 2015. Furthermore I am reviewing for the Journal of Open Source Software.}
  \resumeSubItem{Conferences}{ICASSP, EUSIPCO, DAFx, ISMIR since 2014}
\resumeSubHeadingListEnd

\section{International Collaborations}
\resumeSubHeadingListStart
  \resumeSubItem{Zafar Rafii}{(Gracenote, USA, senior scientist). Collaboration on community service for researchers working on audio source separation. The outcome of this collaboration is \textsc{MUSDB18}, the most popular music corpus for signal processing research, as well as journal papers.}
  \resumeSubItem{Roland Badeau}{(Telecom Paris, France, prof.). Collaboration on probabilistic models for signal processing. This work targeted at proposing new models and optimization methods for robust signal processing and its outcome were conference}
  \resumeSubItem{Antoine Liutkus}{(Inria, France, CRCN). Collaboration on probabilistic models for signal processing and optimal transport, leading to the conference papers. We have also been focused on community service, co-organizing the SiSEC international evaluation, as well as writing overview invited journal papers on music separation.}
  \resumeSubItem{Mark Plumbley}{(Univ. Surrey, UK, prof.). Collaboration on perceptual evaluation for large scale audio processing.}
  \resumeSubItem{Stefan Preihs}{(Univ. of Hannover, Germany). Collaboration on the design and perceptual evaluation of real-time audio processing systems.}
  \resumeSubItem{Yuki Mitsufuji}{(Sony, Japan, Deputy manager of science). Collaboration on a state-of-the-art reference system for music demixing. The outcome of this collaboration is \textsc{open-unmix}, which now serves as the baseline system for other researchers as well as end-users.}
  \resumeSubItem{Fraunhofer IIS}{ I have been involved in many collaborations with researchers from different institutions, where my expertise in deep learning was useful for the design of various audio systems.}
\resumeSubHeadingListEnd

\section{Graduate programs}
 I gave over 120 hours of classes between 2014 and 2016 in the, divided into practical work on computers and tutorials. I also taught distinct subjects in computer science and signal processing:
\resumeSubHeadingListStart
 \resumeSubItem{Guest Lecture (2h)}{2021, Music Source Separation, University of Erlangen-Nürnberg, Germany}
 \resumeSubItem{Introduction to Deep Learning (6h)}{2018, 2019, Master 2, PolyTech Montpellier}
 \resumeSubItem{Reproducible Audio Research Seminar (12h)}{(2016), international master in signal processing, University of Erlangen (Germany)}
 \resumeSubItem{Multimedia Programming 80h)}{2014-2016, level BAC, University of Erlangen (Germany)}
 \resumeSubItem{Statistical Methods for Audio Experiments}{2013-2016, (60h), international master in signal processing, University of Erlangen (Germany).}
\resumeSubHeadingListEnd

\section{Invited Talks/Tutorials/Summer Schools
Schools}
  \resumeSubHeadingListStart
    \resumeSubItem{Invited Talk (1h) at AES Virtual Symposium: Applications of Machine Learning in Audio}{
         "Current Trends in Audio Source Separation". (\href{https://www.youtube.com/watch?v=AB-F2JmI9U4}{Video})}
    \resumeSubItem{Tutorial (3h)}
    {»Music Separation with DNNs: Making It Work«, ISMIR, 2018 (Paris, September 4th, 2018)}
    \resumeSubItem{Invited Talk (1h)}
    {»Deep Learning for Music Unmixing«, Deep Learning: from
      theory to applications, Technicolor (Rennes, September 23th, 2018)}
    \resumeSubItem{Tutorial (3h)}
    {»Music Separation with DNNs: Making It Work«, ISMIR, 2018 (Paris, September 4th, 2018)}
    \resumeSubItem{Presentation}
    {Summer School for Pitch, Music and Associated Pathologies (Lyon, 2014)}
  \resumeSubHeadingListEnd

\section{Prizes}
\resumeSubHeadingListStart
    \resumeSubItem{Global summer PyTorch Hackathon 2020}
        {Winner with DeMask~\href{https://pytorch.org/blog/announcing-the-winners-of-the-2020-global-pytorch-summer-hackathon/}{See PyTorch Website}}
    \resumeSubItem{Global summer PyTorch Hackathon 2019}
        {Winner with Open-Unmix~\href{https://anr.fr/fr/actualites-de-lanr/details/news/open-unmix-un-logiciel-open-source-issu-du-projet-anr-kamoulox-pour-demixer-la-musique/}{See ANR Website}}
\resumeSubHeadingListEnd

% --------PROGRAMMING SKILLS------------
\section{Skills and Tools}
 \resumeSubHeadingListStart
   \resumeSubItem{Programming Languages}
     {Python, Javascript, Matlab, R, C, Java, Julia, SQL}
   \resumeSubItem{DevOps/Deployment Tools}
     {Docker, Kubernetes, Git, Github/Gitlab, AWS, HPC, pytest}
   \resumeSubItem{Machine Learning Frameworks}
     {PyTorch, Tensorflow, Keras, NNabla, Scikit-Learn, statsmodels}
     \resumeSubHeadingListEnd

 % --------PROGRAMMING SKILLS------------
\section{Languages}
\resumeSubHeadingListStart
  \item{
    \textbf{German}{: native}
    \textbf{English}{: fluent (C1)}
    \textbf{French}{: basic (A2)}
  }
\resumeSubHeadingListEnd

\end{document}
